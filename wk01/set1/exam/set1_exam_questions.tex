\documentclass[11pt]{article}
\usepackage[margin=1in]{geometry}
\usepackage{amsmath,amssymb}
\usepackage{enumitem}
\usepackage[T1]{fontenc}

\title{COMP2300 Set 1 Practice Exam}
\author{Topics: Representation, Number Systems, Logic, CMOS Basics}
\date{}

\begin{document}
\maketitle

\noindent\textbf{Time:} 120 minutes \hfill
\textbf{Total:} 100 marks

\vspace{0.5em}
\noindent\textbf{Instructions}
\begin{itemize}[leftmargin=1.5em]
  \item Answer in English.
  \item Part A is multiple-choice (concept-focused).
  \item Part B is computational. Show all working for full credit.
  \item Assume fixed-width arithmetic where width is explicitly given.
\end{itemize}

\section*{Part A: Multiple Choice (30 marks, 3 marks each)}
Choose exactly one option for each question.

\begin{enumerate}[label=\textbf{A\arabic*.}]
  \item In the transformation hierarchy, which is the hardware/software interface?
  \begin{enumerate}[label=(\Alph*)]
    \item Algorithm
    \item ISA
    \item Microarchitecture
    \item Circuits
  \end{enumerate}

  \item Which statement about ISA vs. microarchitecture is correct?
  \begin{enumerate}[label=(\Alph*)]
    \item ISA is how transistors are wired.
    \item Microarchitecture is the programmer-visible instruction definition.
    \item Different microarchitectures can implement the same ISA.
    \item ISA changes every time clock frequency changes.
  \end{enumerate}

  \item Why is binary preferred in digital systems?
  \begin{enumerate}[label=(\Alph*)]
    \item It always uses less memory than decimal.
    \item It simplifies hardware and improves reliability.
    \item It removes the need for abstraction.
    \item It avoids all overflow.
  \end{enumerate}

  \item Which is true for 2's complement representation?
  \begin{enumerate}[label=(\Alph*)]
    \item It has two representations for zero.
    \item It cannot represent negative numbers.
    \item Ordinary binary addition can be used.
    \item Overflow is detected only by carry-out.
  \end{enumerate}

  \item For 2's complement addition, overflow occurs when:
  \begin{enumerate}[label=(\Alph*)]
    \item Carry-out from MSB is 1.
    \item The addends have different signs.
    \item The addends have the same sign and result has opposite sign.
    \item Result is zero.
  \end{enumerate}

  \item Sign extension of an $N$-bit 2's complement value to $M$ bits ($M>N$):
  \begin{enumerate}[label=(\Alph*)]
    \item Always appends zeros.
    \item Always appends ones.
    \item Replicates the sign bit into new high bits.
    \item Inverts all bits.
  \end{enumerate}

  \item In CMOS logic, a valid steady state ideally has:
  \begin{enumerate}[label=(\Alph*)]
    \item Pull-up ON and pull-down ON.
    \item Pull-up OFF and pull-down OFF.
    \item Exactly one of pull-up/pull-down ON.
    \item Both ON if output is 1.
  \end{enumerate}

  \item Which function outputs 1 when inputs are different?
  \begin{enumerate}[label=(\Alph*)]
    \item AND
    \item OR
    \item XOR
    \item XNOR
  \end{enumerate}

  \item A bit mask is primarily used to:
  \begin{enumerate}[label=(\Alph*)]
    \item Increase clock frequency.
    \item Select/clear specific bits.
    \item Convert analog to digital.
    \item Remove sign extension.
  \end{enumerate}

  \item A half adder computes:
  \begin{enumerate}[label=(\Alph*)]
    \item Sum of two bits with carry-in.
    \item Sum and carry for two input bits (no carry-in).
    \item Product and quotient.
    \item Bitwise NOR and NAND.
  \end{enumerate}
\end{enumerate}

\section*{Part B: Computational Problems (70 marks)}
\begin{enumerate}[label=\textbf{B\arabic*.}]
  \item \textbf{Information content and encoding (5 marks)}\\
  (a) How many bits are minimally required to encode 26 symbols?\\
  (b) How many bits are minimally required to encode 100 symbols?

  \item \textbf{Base conversion I (6 marks)}\\
  Convert decimal \(53_{10}\) to binary using any valid method.

  \item \textbf{Base conversion II (6 marks)}\\
  Convert binary \(11010110_2\) to decimal.

  \item \textbf{Binary/hex conversion (6 marks)}\\
  (a) Convert \(1111\,0111_2\) to hexadecimal.\\
  (b) Convert \(D741_{16}\) to binary.

  \item \textbf{Representable ranges (6 marks)}\\
  For \(N=8\), give the min and max for:\\
  (a) Unsigned\\
  (b) Sign/magnitude\\
  (c) 2's complement

  \item \textbf{2's complement encoding/decoding (8 marks)}\\
  Using 8 bits:\\
  (a) Encode \(-18\) in 2's complement.\\
  (b) Decode \(1110\,1011_2\) as a signed 2's complement integer.

  \item \textbf{Unsigned addition and overflow (6 marks)}\\
  Add \(1111_2 + 1111_2\) as 4-bit unsigned values.\\
  Provide 4-bit result and state whether overflow occurs.

  \item \textbf{2's complement addition and overflow (8 marks)}\\
  Use 5-bit 2's complement arithmetic.\\
  (a) \(01001_2 + 01011_2\)\\
  (b) \(10100_2 + 11010_2\)\\
  For each, report binary result, decimal interpretation, and whether overflow occurs.

  \item \textbf{Sign extension (5 marks)}\\
  Sign-extend \(1101_2\) (4-bit 2's complement) to 8 bits, then give its decimal value.

  \item \textbf{Bitwise operations (6 marks)}\\
  Let \(A=00110110_2\), \(B=00001111_2\). Compute:\\
  (a) \(A\ \&\ B\)\\
  (b) \(A\ |\ B\)\\
  (c) \(A\ \oplus\ B\)

  \item \textbf{Bit mask tasks (6 marks)}\\
  Let \(X=10110010_2\).\\
  (a) Extract the lowest 4 bits of \(X\) (show mask and result).\\
  (b) Clear the lowest 2 bits of \(X\) (show mask and result).\\
  (c) Set the highest 2 bits of \(X\) to 1 (show mask and result).

  \item \textbf{Logic/truth table and useful circuits (8 marks)}\\
  (a) For \(Y=(A+B)'\), complete the 2-input truth table (A,B,Y).\\
  (b) A 2:1 MUX has inputs \(I_0=1\), \(I_1=0\), select \(S\). Write \(Y\) in terms of \(S\), then evaluate for \(S=0\) and \(S=1\).\\
  (c) For a half adder with inputs \(A=1, B=1\), compute Sum and Carry.
\end{enumerate}

\end{document}
